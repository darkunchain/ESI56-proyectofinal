\documentclass[stu, 12pt, letterpaper,]{apa7}
\usepackage[utf8]{inputenc}
\usepackage[T1]{fontenc}
\usepackage[spanish]{babel}
\usepackage{times} % Fuente Times New Roman 12pt

% Configuración de márgenes y papel
\usepackage[letterpaper, left=2.54cm, right=2.54cm, top=2.54cm, bottom=2.54cm]{geometry}
% Configuración de espaciado y párrafos
\usepackage{setspace}
\doublespacing% Interlineado doble
\setlength{\parindent}{1.27cm} % Sangría de primera línea
\setlength{\parskip}{0pt} % Sin espacio adicional entre párrafos
\usepackage{ragged2e}
\RaggedRight% Texto alineado a la izquierda sin justificación

% Configuración de tablas y figuras
\usepackage{caption}
\captionsetup[table]{justification=raggedright, singlelinecheck=off, labelsep=space, font={doublespacing}}
\captionsetup[figure]{justification=raggedright, singlelinecheck=off, labelsep=space, font={doublespacing}}

% Configuración de encabezados
\usepackage{titlesec}
\titleformat{\chapter}[display]
  {\normalfont\centering\bfseries\doublespacing}
  {}{0pt}{\large}
\titlespacing*{\chapter}{0pt}{0pt}{0pt} % Ajuste de espaciado para capítulos

% Estilos para los 5 niveles de encabezados
\titleformat{\section}
  {\normalfont\bfseries\doublespacing\centering}
  {}{0pt}{}
\titlespacing*{\section}{0pt}{0pt}{0pt}

\titleformat{\subsection}
  {\normalfont\bfseries\doublespacing\raggedright}
  {}{0pt}{}
\titlespacing*{\subsection}{0pt}{0pt}{0pt}

\titleformat{\subsubsection}
  {\normalfont\bfseries\itshape\doublespacing\raggedright\/}
  {}{0pt}{}
\titlespacing*{\subsubsection}{0pt}{0pt}{0pt}

\newcommand{\nivelcuatro}[1]{
  \noindent
  \hangindent=1.27cm
  \textbf{#1.}\space
}

\newcommand{\nivelcinco}[1]{
  \noindent
  \hangindent=1.27cm
  \textbf{\textit{#1.}}\space
}

% Numeración de páginas
\usepackage{fancyhdr}
\pagestyle{fancy}
\fancyhf{}
\rhead{\thepage}
\renewcommand{\headrulewidth}{0pt}

% Configuración de índice/tabla de contenido
\usepackage[titles]{tocloft}
\renewcommand{\cfttoctitlefont}{\normalfont\bfseries\centering\large}
\renewcommand{\cftaftertoctitle}{\vspace*{\baselineskip}}
\renewcommand{\cftchapleader}{\cftdotfill{\cftdotsep}}
\setlength{\cftbeforechapskip}{0pt}
\setlength{\cftchapindent}{0pt}
\setlength{\cftsecindent}{1.27cm}
\setlength{\cftsubsecindent}{2.54cm}
\renewcommand{\cftchapfont}{\normalfont}
\renewcommand{\cftsecfont}{\normalfont}
\renewcommand{\cftsubsecfont}{\normalfont}

\begin{document}

% Numeración de páginas desde portada
\pagenumbering{arabic}
\setcounter{page}{1}

% Portada
\begin{titlepage}
    \centering
    \vspace*{1cm}
    {\LARGE\textbf{TÍTULO DE LA TESIS}\par}
    \vspace{1.5cm}
    {\large Tesis presentada para optar al título de\par}
    \vspace{2cm}
    {\large\textbf{Tu Nombre}\par}
    \vspace{1cm}
    {\large Bajo la dirección de\par}
    \vspace{2cm}
    {\large Universidad\par}
    {\large Facultad de\par}
    {\large Departamento de\par}
    {\large Ciudad, País\par}
    {\large \today\par}
\end{titlepage}

% Tabla de contenido
\tableofcontents
\thispagestyle{fancy}
\newpage

% Lista de figuras y tablas (opcional)
\listoffigures
\thispagestyle{fancy}
\listoftables
\thispagestyle{fancy}

% Resumen
\chapter*{Resumen}
\addcontentsline{toc}{chapter}{Resumen}
\thispagestyle{fancy}
\noindent
Texto del resumen comenzando al margen izquierdo, sin sangría\ldots

\vspace{0.5cm}
\noindent
\textbf{Palabras clave:} palabra1, palabra2, palabra3

% Abstract
\chapter*{Abstract}
\addcontentsline{toc}{chapter}{Abstract}
\thispagestyle{fancy}
\noindent
English abstract text\ldots

\vspace{0.5cm}
\noindent
\textbf{Keywords:} keyword1, keyword2, keyword3

% Cuerpo principal
\mainmatter% aqui va el cuerpo principal
% Ejemplo de encabezados
\chapter{Introducción} % Nivel 1
Texto introductorio\ldots

\section{Planteamiento del Problema} % Nivel 2
Texto del planteamiento\ldots

\subsection{Justificación} % Nivel 3
Texto de justificación\ldots

\nivelcuatro{Objetivo General} % Nivel 4
Texto que continúa en la misma línea. Este es un ejemplo de cómo se vería un encabezado de nivel 4 según normas APA 7ma edición.

\nivelcinco{Hipótesis} % Nivel 5
Texto que continúa en la misma línea. Este es un ejemplo de cómo se vería un encabezado de nivel 5 según normas APA 7ma edición.

% Ejemplo de tabla
\begin{table}[h]
\centering
\caption{Ejemplo de tabla APA}
\begin{tabular}{ll}
\toprule
Columna 1 & Columna 2 \\
\midrule
Dato 1 & Dato 2 \\
Dato 3 & Dato 4 \\
\bottomrule
\end{tabular}
\end{table}

% Ejemplo de figura
\begin{figure}[h]
\centering
\includegraphics[width=0.5\textwidth]{ejemplo.png}
\caption{Ejemplo de figura APA}\label{fig:ejemplo}% label de la tabla
\end{figure}

% Bibliografía
\backmatter% uso para la bibliografia
\bibliography{referencias}

\end{document}